\newpage
\section{Conclusion}
Au terme de ce cahier des charges, nous pensons avoir fourni une solution viable et correcte afin de répondre aux besoins de système d'informations de l'agence immobilière.\\

Nous avons commencé par réaliser une première analyse des documents fournis par l'agence, afin d'en tirer une ébauche de schéma entité-association. Cette ébauche a par la suite été complétée à l'aide de différents outils. Nous nous sommes servis des diagramme de type use-case afin de définir les grandes fonctionnalités du système. Nous en avons élaboré les scénarios complets afin de vérifier la justesse de notre schéma entité-association et le corriger si nécessaire. Par la suite, nous avons traduit ces scénarios en diagrammes d'activité, représentant les interactions entre les acteurs des scénarios, le système, et les conditions rencontrées en cours de traitement. Nos avons enfin complété notre ébauche à l'aide des diagrammes UML. Nous aurions aimé rendre nos entités encore plus précises à l'aide des diagramme de séquence, mais nous avons manqué de temps pour l'élaboration de ceux-ci. Nous nous sommes donc contentés de ce qui avait été réalisé auparavant.\\

Au final, nous sommes arrivés a un schéma entité-association complété par les différents outils mis à notre disposition, réalisé sous DB-Main. Ce schéma association répond aux attentes de l'agence immobilière pour la réalisation de leur système d'informations. Ce schéma final a enfin été traduit en schéma relationnel. Celui-ci représente la base de données à implémenter pour l'agence immobilière, sous forme des diverses tables, ainsi que les tables ajoutées par les associations, les clés primaires, et les clés étrangères liant le tout.\\

Nous espérons, au travers de ce document, avoir répondu à la demande de système d'informations de la part de l'agence immobilière. Nous osons croire que l'analyse que nous avons fournie est suffisamment complète et satisfera notre client.