\section{Introduction}
Ce cahier des charges est destiné à répondre à la demande de mise en place d'un système d'information pour une agence immobilière locale. Nous commençons ce document en présentant le cadre dans lequel ce cahier des charges a été demandé. Nous passons ensuite à une première analyse de la problématique sous forme d'une ébauche pouvant amener à un schéma entité-association, devant être complété par après. Nous fournissons bien entendu des explications justifiants nos choix. Par la suite, nous présentons la manière dont nous avons utilisé différents outils mis à notre disposition afin d'obtenir un prototype de base de données le plus efficace possible. Nous utilisons les outils diagramme use-case, diagramme d'activité, diagramme de classes UML, et diagramme de séquence. Nous expliquons enfin le résultat auquel nous sommes parvenus, et que nous considérons comme le plus efficace pour l'implémentation du système d'informations demandé.