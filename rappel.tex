\section{Présentation du cadre du projet}
Le travail qui nous a été demandé de réaliser a pour client une agence immobilière locale, souhaitant moderniser son infrastructure. Celle-ci a pour vocation de se placer en tant qu'intermédiaire entre des propriétaires de biens immobiliers et d'éventuels locataires ou acheteurs. Ces biens peuvent être soit à louer, soit à acheter, et de différents types : 
\begin{itemize}
	\item maison,
	\item appartement,
	\item terrain,
	\item emplacement,
	\item entrepôt,
	\item etc.\\
\end{itemize}

Afin de réaliser leur mission, cette agence immobilière s'aide de divers services. Ces services sont internes à l'agence, et ont pour but de permettre une meilleure organisation de cette dernière. Les services permettant cette optimisation de la gestion sont par exemple :
\begin{itemize}
\item Le service d'enregistrement des demandes, qui s'occupe de l'enregistrement du bien d'un propriétaire, mais également la gestion des demandes de biens de la part d'un potentiel acheteur ou locataire. De ce fait, ce service gère également les contrats.
\item Le service des visites, qui s'occupe de mettre en place les plannings des visites, aussi bien pour les clients que pour les agents.
\item Le service des statistiques, qui, comme son nom l'indique, s'occupe des statistiques concernant les demandes des clients par rapport aux biens. \\
\end{itemize}

Parallèlement à ces différentes aides à l'organisation, l'agence immobilière à dorénavant besoin d'un système d'informations. Ce système d'informations supplémentaire sera implémenté à l'aide de la base de données décrite dans ce cahier des charges.\\

Afin que nous puissions correctement concevoir le système demandé, l'agence immobilière a mis à notre disposition un certain nombre d'informations. Celles-ci nous permettent de mieux comprendre les différents aspects du métier d'agent immobilier, et ainsi pouvoir conceptualiser au mieux la demande. Les informations nous ont été remises par écrit. Nous disposons donc d'un cas papier, ainsi que d'un contact avec le client, pour appréhender au mieux la situation lors de l'analyse qui va suivre.